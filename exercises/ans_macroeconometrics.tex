\begin{Solution}{1}
	DSGE models use modern macroeconomic theory to explain and predict co-movements of aggregate time series. DSGE models start from what we call the micro-foundations of macroeconomics (i.e. to be consistent with the underlying behavior of economic agents), with a heart based on the rational expectation forward-looking economic behavior of agents. In reality all macro variables are related to each other, either directly or indirectly, so there is no \enquote{cetribus paribus}, but a dynamic stochastic general equilibrium system.
	\begin{itemize}
		\item General Equilibrium (GE): equations must always hold. Short-run: decisions, quantities and prices adjust such that equations are full-filled. Long-run: steady state, i.e. a condition or situation where variables do not change their value (e.g. balanced-growth path where the rate of growth is constant).
		\item Stochastic (S): disturbances (or shocks) make the system deviate from its steady state, we get business cycles or, more general, a data-generating process
		\item Dynamic (D): Agents are forward-looking and solve intertemporal optimization problems. When a disturbance hits the economy, macroeconomic variables do not return to equilibrium instantaneously, but change very slowly over time, producing complex reactions. Furthermore, some decisions like investment or saving only make sense in a dynamic context. We can analyze and quantify the effects after (i) a temporary shock: how does the economy return to its steady state, or (ii) a permanent shock: how does the economy move to a new steady state.
	\end{itemize}
	Basic structure:
	$$ E_t \left[f(y_{t+1}, y_t, y_{t-1},u_t)\right]=0$$
	where $E_t$ is the expectation operator with information conditional up to and including period $t$, $y_t$ is a vector of endogenous variables at time $t$, $u_t$ a vector of exogenous shocks or random disturbances with proper density functions. $f(\cdot)$ is what we call economic theory. \textbf{First key challenge:} value of endogenous variables in a given period of time depend on its future expected value. We need dynamic programming techniques to find the optimality conditions which define the economic behavior of the agents.

	The solution to this system is a decision function:
	$$y_t = g(y_{t-1},u_t)$$ \textbf{Second key challenge}: DSGE models cannot be solved analytically, except for some very simple and unrealistic examples. We have to resort to numerical methods and a computer to find an approximated solution.

	Once the theoretical model and solution is at hands, the next step is the \textbf{third key challenge}: application to the data. The usual procedure consists in the calibration of the parameters of the model using previous information or matching some key ratios or moments provided by the data, or more recently, form the estimation of the parameters using maximum likelihood, Bayesian techniques, indirect inference, or general method of moments.


	
\end{Solution}
\begin{Solution}{2}
	Focus on behavior of three main types of economic agents or sectors:
	\begin{itemize}
		\item Households: benefit from private consumption, leisure and possibly other things like money holdings or state services; subject to a budget constraint in which they finance their expenditures via (utility-reducing) work, renting capital and buying (government) bonds $\hookrightarrow$ maximization of utility
		\item Firms produce a variety of products with the help of rented equipment (capital) and labor. They (possibly) have market power over their product and are responsible for the design, manufacture and price of their products. $\hookrightarrow$
		cost minimization or profit maximization
		\item Monetary policy follows a feedback rule, so-called Taylor rule, for instance: nominal interest rate reacts to deviations of the current (or lagged) inflation rate from its target and of current output from potential output
		\item Fiscal policy (the government) collects taxes from households and companies in order to finance government expenditures (possibly utility-enhancing) and government investment (possibly productivity-enhancing). In addition, the government can issue debt securities.
		\end{itemize}
	Also other sectors possible: financial sector, foreign sector, etc. Equilibrium results from the combination of economic decisions taken by all economic agents.

	\begin{itemize}
		\item Canonical neoclassical model (RBC model): reduce economy to the interaction of just one (representative) consumer/household and one (representative) firm. Representative household takes decisions in terms of how much to consume (save) and how much time is devoted to work (leisure). Representative firm decides how much it will produce. Equilibrium of the economy will be defined by a situation in which all decisions taken by all economic agents are compatible and feasible. One can show that business cycles can be generated by one special disturbance: total factor productivity or neutral technological shock; hence, model generates real business cycles without nominal frictions.
		\item Scale of DSGE models has grown over time with incorporation of a large number of features. To name a few: consumption habit formation, nominal and real rigidities, non-Ricardian agents, investment adjustment costs, investment-specific technological change, taxes, public spending, public capital, human capital, household production, imperfect competition, monetary union, steady state unemployment etc.
		\item New-Keynesian models have the same foundations as New-Classical general equilibrium models, but incorporate different types of rigidities in the economy. Whereas new classical DSGE models are constructed on the basis of a perfect competition environment, New-Keynesian models include additional elements to the basic model such as imperfect competitions, existence of adjustment costs in investment process, liquidity constraints or rigidities in the determination of prices and wages.
	\end{itemize}
	
\end{Solution}
\begin{Solution}{3}
	The degree of realism offered by an economic model is not a goal to be pursued by macroeconomists, but rather the model's usefulness in explaining macroeconomic reality. General strategy is the construction of formal structures through equations that reflect the interrelationships between the different economic variables. These simplified structures is what we call a model. The essential question is not that these theoretical constructions are realistic descriptions of the economy, but that they are able to explain the dynamics observed in the economy. Therefore, it is not possible to reject a model ex ante because it is based on assumptions that we believe not too realistic. Rather, the validations must be based on the usefulness of these models to explain reality, and whether they are more useful than other models.

	Regarding the assumption that the lifetime of economic agents is assumed to be infinite: We know that in reality consumers, firms and governments have finite life. However, in our models and to be more precise, we assume that firms and governments both use the infinite time as \textbf{the reference period for taking economic decisions}. This is not unrealistic: no government thinks it will cease to exist at some point in the future and no entrepreneur takes decisions based on the idea that the firm will go bankrupts sometime in the future. For consumers this is not so realistic, however, we may weaken this assumption, and think about families, dynasties or households rather than consumers, then the infinite time planning horizon assumption is feasible. Of course, to study the finite life cycle of an agent (school-work-retirement), the so-called Overlapping Generations (OLG) framework is more useful.
	
\end{Solution}
\begin{Solution}{1}
			The transversality condition for an infinite horizon dynamic optimization problem is the boundary condition determining a solution to the problem's first-order conditions together with the initial condition. The transversality condition requires the present value of the state variables (here $K_t$ and $A_t$) to converge to zero as the planning horizon recedes towards infinity. The first-order and transversality conditions are sufficient to identify an optimum in a concave optimization problem. Given an optimal path, the necessity of the transversality condition reflects the impossibility of finding an alternative feasible path for which each state variable deviates from the optimum at each time and increases discounted utility.
		
\end{Solution}
\begin{Solution}{2}
		Due to our assumptions , we will not have corner solutions and can neglect the non-negativity constraints. Due to the transversality condition and the concave optimization problem, we only need to focus on the first-order conditions.
		The Lagrangian for the household problem is
		\begin{align*}
		L = E_t\sum_{j=0}^{\infty}\beta^j&\left\{\gamma \log(C_{t+j}) + \psi \log(1-l_{t+j})\right.\\
		&+\lambda_{t+j}\left(W_{t+j} L_{t+j} + R_{t+j} K_{t+j} - C_{t+j} - I_{t+j}\right)\\
		&+ \left. \mu_{t+j} \left((1-\delta)K_{t+j} + I_{t+j} - K_{t+j+1}\right)\right\}
		\end{align*}
		Note that the problem is not to choose $\{C_t,I_t,L_t,K_{t+1}\}_{t=0}^\infty$ all at once in an open-loop policy, but to choose these variables sequentially given the information at time $t$ in a closed-loop policy, i.e. at period $t$ decision rules for $\{C_t,I_t,L_t,K_{t+1}\}$ given the information set at period $t$; at period $t+1$ decision rules for $\{C_{t+1},I_{t+1},L_{t+1},K_{t+2}\}$ given the information set at period $t+1$.

		The first-order condition w.r.t. $C_t$ is given by
		\begin{align*}
		\frac{\partial L}{\partial C_{t}} &= E_t \left(\gamma C_t^{-1}-\lambda_{t}\right) = 0\\
		\Leftrightarrow \lambda_{t} &= \gamma C_{t}^{-1} & (I)
		\end{align*}
		The first-order condition w.r.t. $L_t$ is given by
		\begin{align*}
		\frac{\partial L}{\partial L_{t}} &= E_t \left(\frac{-\psi}{1-L_{t}}+\lambda_{t} W_{t}\right) = 0\\
		\Leftrightarrow \lambda_{t} W_{t} &= \frac{\psi}{1-L_{t}} &(II)
		\end{align*}
		The first-order condition w.r.t. $I_{t}$ is given by
		\begin{align*}
		\frac{\partial L}{\partial I_{t}} &= E_t \beta^j \left(-\lambda_{t} + \mu_{t}\right) = 0\\
		\Leftrightarrow \lambda_{t} &= \mu_{t} & (III)
		\end{align*}
		The first-order condition w.r.t. $K_{t+1}$ is given by
		\begin{align*}
		\frac{\partial L}{\partial K_{t+1}} &= E_t (-\mu_{t}) +
		E_t \beta \left(\lambda_{t+1}R_{t+1}+\mu_{t+1}(1-\delta)\right) = 0\\
		\Leftrightarrow \mu_{t} &= E_t \beta(\mu_{t+1}(1-\delta)+\lambda_{t+1}R_{t+1}) & (IV)
		\end{align*}

		(I) and (III) in (IV) yields
		\begin{align*}
		\underbrace{\gamma C_t^{-1}}_{U_t^c} &= \beta E_t \underbrace{\gamma C_{t+1}^{-1}}_{U_{t+1}^c}\left(1-\delta + R_{t+1}\right)
		\end{align*}
		This is the Euler equation of intertemporal optimality. It reflects the trade-off between consumption and savings. If the household saves a (marginal) unit of consumption, she can consume the gross rate of return on capital, i.e. $(1-\delta+R_{t+1})$ units, in the following period. The marginal utility of consuming today is equal to $U_t^c$, whereas consuming tomorrow has expected utility $E_t(U_{t+1}^c)$. Discounting expected marginal utility with $\beta$ the household mist be indifferent between both choices in the optimum.

		(I) in (II) yields:
		\begin{align*}
		W_t = -\frac{\frac{-\psi}{1-L_t}}{\gamma C_t^{-1}} \equiv - \frac{U_t^l}{U_t^c}
		\end{align*}
		This equation reflects intratemporal optimality, particularly, the optimal choice for labor supply: the real wage must be equal to the marginal rate of substitution between labor and consumption.
		
\end{Solution}
\begin{Solution}{3}
		First, we note that even though firms maximize expected profits it is actually a static problem. That is, the objective is to maximize profits
		\begin{align*}
			\Pi_t = A_t K_t^\alpha L_t^{1-\alpha} - W_t L_t - R_t K_t
		\end{align*}
		The first-order condition w.r.t. $L_{t}$ is given by
		\begin{align*}
		\frac{\partial \Pi_t}{\partial L_{t}} &= (1-\alpha) A_t K_t^\alpha L_t^{-\alpha} - W_t = 0\\
		\Leftrightarrow W_t &= (1-\alpha) A_t K_t^\alpha L_t^{-\alpha} = f_L = (1-\alpha) \frac{Y_t}{L_t}
		\end{align*}
		The real wage must be equal to the marginal product of labor. Due to the Cobb-Douglas production function it is a constant proportion $(1-\alpha)$ of the ratio of total output and labor. This is the labor demand function.

		The first-order condition w.r.t. $K_{t}$ is given by
		\begin{align*}
		\frac{\partial \Pi_t}{\partial K_{t}} &= \alpha A_t K_t^{\alpha-1} L_t^{1-\alpha} - R_t = 0\\
		\Leftrightarrow R_t &= \alpha A_t K_t^{\alpha-1} L_t^{1-\alpha} = f_K = \alpha \frac{Y_t}{K_t}
		\end{align*}
		The real interest rate must be equal to the marginal product of capital. Due to the Cobb-Douglas production function it is a constant proportion $\alpha$ of the ratio of total output and capital. This is the capital demand function.
		
\end{Solution}
\begin{Solution}{4}
		First, consider the steady state value of technology:
		$$\log\bar{A}=\rho_A \log\bar{A} + 0 \Leftrightarrow \log\bar{A} = 0 \Leftrightarrow \bar{A} = 1$$
		The Euler equation in steady state becomes:
		\begin{align*}
			\bar{U}^C &= \beta \bar{U}^C(1-\delta+\bar{R})\\
			\Leftrightarrow \bar{R} &= \frac{1}{\beta} + \delta - 1
		\end{align*}
		Now we will provide recursively closed-form expressions for all variables in relation to steady state labor.
		\begin{itemize}
		\item The firms demand for capital in steady state becomes
		\begin{align*}
			\bar{R} &= \alpha \bar{A} \bar{K}^{\alpha-1}\bar{L}^{1-\alpha}\\
			\Leftrightarrow \frac{\bar{K}}{\bar{L}} &= \left(\frac{\alpha \bar{A}}{\bar{R}}\right)^{\frac{1}{1-\alpha}}
		\end{align*}
		\item The firms demand for labor in steady state becomes:
		\begin{align*}
			W =(1-\alpha) \bar{A}\bar{K}^\alpha \bar{L}^{-\alpha} = (1-\alpha)\bar{A} \left(\frac{\bar{K}}{\bar{L}}\right)^\alpha
		\end{align*}
		\item The law of motion for capital in steady state implies
		\begin{align*}
			\frac{\bar{I}}{\bar{L}} &= \delta\frac{\bar{K}}{\bar{L}}
		\end{align*}
		\item The production function in steady state becomes
		\begin{align*}
			\frac{\bar{Y}}{\bar{L}} = \bar{A} \left(\frac{\bar{K}}{\bar{L}}\right)^\alpha
		\end{align*}
		\item The clearing of the goods market in steady state implies
		\begin{align*}
		\frac{\bar{C}}{\bar{L}} = \frac{\bar{Y}}{\bar{L}} - \frac{\bar{I}}{\bar{L}}
		\end{align*}
		\end{itemize}
		Now, we need to derive steady state labor from the equilibrium on the labor market. Due to the log-utility, we can derive a closed-form expression:
		\begin{align*}
			\psi \frac{1}{1-\bar{L}} &= \gamma \bar{C}^{-1} W\\
			\Leftrightarrow \psi \frac{\bar{L}}{1-\bar{L}} &= \gamma \left(\frac{\bar{C}}{\bar{L}}\right)^{-1} W\\
			\Leftrightarrow \bar{L} &= (1-\bar{L})\frac{\gamma}{\psi} \left(\frac{\bar{C}}{\bar{L}}\right)^{-1} W\\
			\Leftrightarrow \bar{L} &= \frac{\frac{\gamma}{\psi} \left(\frac{\bar{C}}{\bar{L}}\right)^{-1} W}{1+\frac{\gamma}{\psi} \left(\frac{\bar{C}}{\bar{L}}\right)^{-1} W}\\
		\end{align*}
		Lastly, it is straigforward to compute the remaining steady state values, i.e.
		\begin{align*}
		\bar{C} = \frac{\bar{C}}{\bar{L}}\bar{L},\qquad
		\bar{I} = \frac{\bar{I}}{\bar{L}}\bar{L},\qquad
		\bar{K} = \frac{\bar{K}}{\bar{L}}\bar{L},	\qquad
		\bar{Y} = \frac{\bar{Y}}{\bar{L}}\bar{L}
		\end{align*}
		
\end{Solution}
\begin{Solution}{5}
 General hints: construct and parameterize the model such, that it corresponds to certain properties of the true economy. One often uses steady state characteristics for choosing the parameters in accordance with observed data. For instance, long-run averages (wages, working-hours, interest rates, inflation, consumption-shares, government-spending-ratios, etc.) are used to fix steady state values of the endogenous variables, which implies values for the parameters. You can use also use micro-studies, however, one has to be careful about the aggregation!

		We will focus on OECD countries and discuss one \enquote{possible} way to calibrate the model parameters (there are many other ways):
		\begin{itemize}
			\item[$\boldsymbol{\alpha}$] productivity parameter of capital. Due to the Cobb Douglas production function thus should be equal to the proportion of capital income to total income of economy. So, one looks inside the national accounts for OECD countries and sets $\alpha$ to 1 minus the share of labor income over total income. For most OECD countries this implies a range of 0.25 to 0.35.
			\item[$\boldsymbol{\beta}$] subjective intertemporal preference rate of households: it is the value of future utility in relation to present utility. Usually takes a value slightly less than unity, indicating that agents discount the future. For quarterly data, we typically set it around 0.99. A better way: fix this parameter by making use of the Euler equation in steady state: $\beta = \frac{1}{\bar{R}+1-\delta}$ where $\bar{R}=\alpha \frac{\bar{Y}}{\bar{K}}$. Looking at OECD data one usually finds that average capital productivity $\bar{K}/\bar{Y}$ is in the range of $9$ to $10$.
			\item[$\boldsymbol{\delta}$] depreciation rate of capital stock. For quarterly data the literature uses values in the range of 0.02 to 0.03. A better way: use steady state implication that $\delta=\frac{\bar{I}}{\bar{K}}=\frac{\bar{I/Y}}{\bar{K/Y}}$. For OECD data one usually finds that average ratio of investment to output, $\bar{I}/\bar{Y}$, is around 0.25.
			\item[$\boldsymbol{\gamma}$ and $\boldsymbol{\psi}$] individual's preference regarding consumption and leisure. Often a certain interpretation in terms of elasticities of substitutions is possible. Here we can make use of the First-Order-Conditions in steady state, i.e.
			$$\frac{\psi}{\gamma} = \bar{W}\frac{(1-\bar{L})}{\bar{C}}= (1-\alpha)\left(\frac{\bar{K}}{\bar{L}}\right)^\alpha\frac{(1-\bar{L})}{\bar{C}} = (1-\alpha)\left(\frac{\bar{K}}{\bar{L}}\right)^\alpha\frac{\frac{1}{\bar{L}}(1-\bar{L})}{\frac{\bar{C}}{\bar{L}}}$$
			and noting that $\bar{C}/\bar{L}$ as well as $\bar{K}/\bar{L}$ are given in terms of already calibrated parameters (see steady state computations). Therefore, one possible way is to normalize one of the parameters to unity (e.g. $\gamma=1$) and calibrate the other one in terms of steady state ratios for which we would only require to calibrate steady state hours worked $\bar{L}$. Note that labor time is normalized and usually corresponds to 8 hours a day, i.e. $\bar{L}=1/3$.
			\item[$\boldsymbol{\rho_A}$ and $\boldsymbol{\sigma_A}$] parameters of process for total factor productivity. These can be estimated based on a regression of the Solow Residual, i.e. production function residuals. $\rho_A$ is mostly set above 0.9 to reflect persistence of the technological process and $\sigma_A$ around $0.6$ in the simple RBC model. Another way would be to try different values for $\sigma_A$ and then try to match the shape of impulse-response-functions of corresponding (S)VAR models.
		\end{itemize}
		
\end{Solution}
\begin{Solution}{6}
		In the mod file, set \texttt{logutility = 1} and for (a) \texttt{steadystatemethod = 2} and for (b) \texttt{steadystatemethod = 1}. For (b) make sure that \texttt{RBC\_leisure\_steadystate.m} is not in the same folder otherwise initval won't be effective.
		%\lstinputlisting{../progs/Dynare/RBC_leisure/RBC_leisure.mod}
		
\end{Solution}
\begin{Solution}{7}
	For the first-order conditions of the household we know use
	\begin{align*}
	U_t^C &= \gamma C_t^{-\eta_C}\\
	U_t^L &= - \psi (1-L_t)^{-\eta_L}
	\end{align*}
	The steady state for labor changes to
	\begin{align*}
		W \gamma C^{-\eta_C} &= \psi(1-L)^{-\eta_L}\\
		W  \left(\frac{C}{L}\right)^{-\eta_C} &= \psi(1-L)^{-\eta_L}L^{\eta_C}
	\end{align*}
	This cannot be solved for $L$ in closed-form. Rather, we need to condition on the values of the parameters and use an numerical optimizer to solve for $L$ as is done in the file \texttt{RBC\_leisure\_steadystate.m}:
	%\lstinputlisting{../progs/Dynare/RBC_leisure/RBC_leisure_steadystate.m}
	Copy \texttt{RBC\_leisure\_steadystate.m} into the same folder as \texttt{RBC\_leisure.mod}. Set \texttt{logutility = 0} and \texttt{steadystatemethod = 0}. Run the mod file with DYNARE to compute the steady state. Note that in the case of log utility the external function uses the closed-form expression to output steady state labor. So this file is most general.
	
\end{Solution}
\begin{Solution}{1}
		Due to our assumptions , we will not have corner solutions and can neglect the non-negativity constraints. Due to the transversality condition and the concave optimization problem, we only need to focus on the first-order conditions.
		The Lagrangian for the household problem is
		\begin{align*}
		L = E_t\sum_{j=0}^{\infty}\beta^j\left\{\log(C_{t+j}) + \lambda_{t+j} \left(A_{t+j}K_{t+j}^\alpha -C_t - K_{t+j+1}\right)\right\}
		\end{align*}
		Note that the problem is not to choose $\{C_t,K_{t+1}\}_{t=0}^\infty$ all at once in an open-loop policy, but to choose these variables sequentially given the information at time $t$ in a closed-loop policy.

		The first-order condition w.r.t. $C_t$ is given by
		\begin{align*}
		\frac{\partial L}{\partial C_{t}} &= E_t \left(C_t^{-1}-\lambda_{t}\right) = 0\\
		\Leftrightarrow \lambda_{t} &= C_{t}^{-1} & (I)
		\end{align*}
		The first-order condition w.r.t. $K_{t+1}$ is given by
		\begin{align*}
		\frac{\partial L}{\partial K_{t+1}} &= E_t (-\lambda_{t}) +
		E_t \beta \left(\lambda_{t+1}\alpha A_{t+1} K_{t+1}^{\alpha-1}\right) = 0\\
		\Leftrightarrow \lambda_{t} &= \alpha\beta E_t \lambda_{t+1}A_{t+1} K_{t+1}^{\alpha-1} & (II)
		\end{align*}

		(I) and (II) yields
		\begin{align*}
		C_t^{-1} = \alpha\beta E_t C_{t+1}^{-1} A_{t+1} K_{t+1}^{\alpha-1}
		\end{align*}
	
\end{Solution}
\begin{Solution}{2}
		First, consider the steady value of technology:
		$$\log\bar{A}=\rho_A \log\bar{A} + 0 \Leftrightarrow \log\bar{A} = 0 \Leftrightarrow \bar{A} = 1$$
		The Euler equation in steady state becomes:
		\begin{align*}
		\bar{K} = (\alpha \beta \bar{A})^{\frac{1}{1-\alpha}}
		\end{align*}
	
\end{Solution}
\begin{Solution}{3}
	Inserting the guessed policy function for $C_t$ inside the the capital accumulation equation yields:
	\begin{align*}
	K_{t+1} = A_{t}K_{t}^\alpha - g_C A_t K_t^\alpha = (1-g_C) A_t K_t^\alpha
	\end{align*}
	Therefore, $h_K=(1-g_C)$. Once we derive $g_C$, we get  $h_K$.

	Inserting the guessed policy function for $C_t$ inside the Euler equation yields
	\begin{align*}
	\frac{1}{C_t} = \alpha \beta E_t \frac{1}{C_{t+1}}A_{t+1} K_{t+1}^{\alpha-1}\\
	\frac{1}{g_C A_t K_t^\alpha} = \alpha \beta E_t \frac{1}{g_C A_{t+1} K_{t+1}^\alpha}A_{t+1} K_{t+1}^{\alpha-1}\\
	A_t K_t^\alpha = \frac{1}{\alpha \beta} E_t K_{t+1}
	\end{align*}
	Inserting the decision rule for capital:
	\begin{align*}
		A_t K_t^\alpha = \frac{1}{\alpha \beta} (1-g_C)A_t K_t^\alpha\\
		\Leftrightarrow g_C = (1-\alpha \beta)
	\end{align*}
	Thus the policy function for $C_t$ is
	$$ C_t = (1-\alpha\beta) A_t K_t^\alpha$$
	and for $K_{t+1}$:
	$$ K_{t+1} = \alpha \beta A_t K_t^\alpha$$
	In summary we have found analytically the policy functions. This will not be possible for other DSGE models and we have to rely on numerical methods to approximate the highly nonlinear functions $g$ and $h$.
	
\end{Solution}
\begin{Solution}{4}
~
	\lstinputlisting{../progs/Dynare/BrockMirman/BrockMirman.mod}
	
\end{Solution}
\begin{Solution}{5}
	Set \texttt{comparison = 1} and then run the above mod file with Dynare. You see that the first-order approximation of the true solution is quite accurate only when we are in the vicinity of the steady state. For technology the first-order approximation is exactly the true decision function, because the first-order approximation is equal to a log-linearization. In the IRFs we understate the effect of the technology shock.
	
\end{Solution}
